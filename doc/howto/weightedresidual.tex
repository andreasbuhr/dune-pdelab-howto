%%%%%%%%%%%%%%%%%%%%%%%%%%%%%%%%%%%%%%%%%%%%%%%%%%%%%%%%%%%%
%%%%%%%%%%%%%%%%%%%%%%%%%%%%%%%%%%%%%%%%%%%%%%%%%%%%%%%%%%%%
%%%%%%%%%%%%%%%%%%%%%%%%%%%%%%%%%%%%%%%%%%%%%%%%%%%%%%%%%%%%
\section{Weighted Residual Formulation}
%%%%%%%%%%%%%%%%%%%%%%%%%%%%%%%%%%%%%%%%%%%%%%%%%%%%%%%%%%%%
%%%%%%%%%%%%%%%%%%%%%%%%%%%%%%%%%%%%%%%%%%%%%%%%%%%%%%%%%%%%
%%%%%%%%%%%%%%%%%%%%%%%%%%%%%%%%%%%%%%%%%%%%%%%%%%%%%%%%%%%%

\subsection{Abstract Formulation}

When we go about to solve a PDE problem we need a general idea of what
a PDE problem is. We will develop such an idea in this section.


We now give some concrete examples.


\subsection{Properties of the Residual Form}


\subsection{Time-dependent Problems}

What about time-dependent problems?

\begin{frame}
\frametitle<presentation>{Time-dependent problems}
After semidiscretization in space we obtain a problem of the form
\begin{equation*}
u_h(t)\in U_h : \quad \frac{\partial}{\partial t} m_h(u_h(t),v;t) + q_h(u_h(t),v;t)
= 0 \qquad \forall v\in V_h, t\in\Sigma. 
\end{equation*}
Implicit Euler (e.g.) then leads to: Find $u_h^{k+1}\in U_h$ s.t.
\begin{equation*}
\underbrace{m_h(u_h^{k+1},v;t^{k+1}) - m_h(u_h^{k},v;t^k) + \Delta
t^{k}q_h(u_h^{k+1},v;t^{k+1})}_{r_h^\text{IE}(u_h,v)}  = 0
\quad v\in V_h.
\end{equation*}

Explicit Euler (e.g.) leads to: Find  $u_h^{k+1}\in U_h$ s.t.
\begin{equation*}
\underbrace{m_h(u_h^{k+1},v;t^{k+1}) - m_h(u_h^{k},v;t^k) + \Delta
t^{k}q_h(u_h^{k},v;t^{k})}_{r_h^\text{EE}(u_h,v)}  = 0
\quad v\in V_h.
\end{equation*}
Higher-order time discretizations lead to similar problems.

Explicit schemes may lead to easily invertible algebraic systems.
\end{frame}

\cleardoublepage
